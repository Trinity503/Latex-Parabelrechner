\documentclass{article}
\usepackage{pgf}
\usepackage{pgfplots}
\pgfplotsset{compat=1.15}
\usepackage{mathrsfs}
\usetikzlibrary{arrows}
\pagestyle{empty}
\usepackage[left=2.5cm,top=2cm,right=2.5cm,bottom=2cm,bindingoffset=0.5cm]{geometry}


\setlength\parindent{0pt}
%Die Werte der Parabel lauten
\pgfmathsetmacro{\a}{4}
\pgfmathsetmacro{\b}{3}
\pgfmathsetmacro{\c}{-2}
\pgfmathsetmacro{\xs}{-\b/(2*\a)}
\pgfmathsetmacro{\ys}{\c-\b*\b/(4*\a)}

%Die Werte der Gerade lauten
\pgfmathsetmacro{\m}{-3}
\pgfmathsetmacro{\t}{1}

\begin{document}
\begin{Huge}
Parabelrechner
\end{Huge}
\\ \\

Die Parabel hat die Gleichung: $y = \a x^2 + \b x + \c$\\
Die Parabel hat die Gleichung in der Scheitelform:$y = \a (x-\xs)^2+\ys$\\ \\
Die Gerade hat die Gleichung:  $y= \m x+ \t$.\\

Der Scheitelpunkt der Parabel ist: $S(\xs | \ys)$ \\

Die Nullstelle der Parabel ist bei: $x_1= \pgfmathparse{
(-\b-sqrt(\b^2-4*\a*\c))/(2*\a)
}\pgfmathresult \wedge x_2 = \pgfmathparse{
(-\b+sqrt(\b^2-4*\a*\c))/(2*\a)
)}\pgfmathresult$ \\


Die Nullstelle der Gerade ist: $x= \pgfmathparse{-\t/\m}\pgfmathresult $ \\

Die Differenz $y_P - y_g$ ist: $\a x^2+ \pgfmathparse{\b-\m}\pgfmathresult x + \pgfmathparse{\c-\t}\pgfmathresult $ \\

Die Differenz $y_g - y_p$ ist: $-\a x^2+ \pgfmathparse{\m-\b}\pgfmathresult x + \pgfmathparse{\t-\c}\pgfmathresult $ \\


Die Schnittpunkte von Parabel und Gerade liegen bei: 
$S_1(\pgfmathparse{
(-(\b-\m)-sqrt((\b-\m)^2-4*\a*(\c-\t)))/(2*\a)
}\pgfmathresult | \pgfmathparse{\m*
(-(\b-\m)-sqrt((\b-\m)^2-4*\a*(\c-\t)))/(2*\a)+\t
}\pgfmathresult )$ und
$S_2(\pgfmathparse{
(-(\b-\m)+sqrt((\b-\m)^2-4*\a*(\c-\t)))/(2*\a)
}\pgfmathresult | \pgfmathparse{\m*
(-(\b-\m)+sqrt((\b-\m)^2-4*\a*(\c-\t)))/(2*\a)+\t
}\pgfmathresult )$ \\
\\ \\

\begin{tikzpicture}[line cap=round,line join=round,>=triangle 45,x=1.0cm,y=1.0cm]
\begin{axis}[
x=1.0cm,y=1.0cm,
axis lines=middle,
ymajorgrids=true,
xmajorgrids=true,
xmin=-5.3000000000000025,
xmax=6.3,
ymin=-6.679999999999995,
ymax=6.299999999999998,
xtick={-5.0,-4.0,...,6.0},
ytick={-6.0,-5.0,...,6.0},]
\clip(-5.3,-6.68) rectangle (6.3,6.3);
\draw [samples=50,rotate around={180.:(\xs,\ys)},xshift=\xs cm,yshift=\ys cm,line width=1.pt,domain=-2.25:2.25)] plot (\x,{(\x)^2/2/(-1/\a});
\draw [line width=1.pt,domain=-5.3:6.3] plot(\x,{(\t \m*\x)/1.});
\end{axis}
\end{tikzpicture}



\end{document}
